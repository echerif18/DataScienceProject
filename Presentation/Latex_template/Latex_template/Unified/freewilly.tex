\documentclass{beamer}

\usepackage{spot}
\usepackage{graphicx}
%\usepackage{subfig}
\usepackage{color,soul}
%=================================================
% packages and new commands
%=================================================
%\usepackage[ruled, linesnumbered, vlined]{algorithm2e}
\usepackage{epsfig, subfigure, amssymb, multirow}

\usepackage{amsmath}
%\usepackage[noend]{algpseudocode}
\usepackage{algorithm, algorithmic}

\newcommand*{\superscript}[1]{\ensuremath{^{\rm #1}}}
\newcommand*{\subscript}[1]{\ensuremath{_{\rm #1}}}

%\usepackage[ruled,vlined]{algorithm2e}
\usepackage{tabulary}
\usepackage{amsmath}
\usepackage{ amssymb }
\usepackage{textpos}
\usepackage{array}
\usepackage{color}
\usepackage{tabularx} 
\usepackage{amsmath}
\usepackage{listings}
\usetheme{Madrid}


%\usepackage{algorithm, algorithmic}

 % for 'tabularx' environment
%\useapackage{ragged2e} % for \Centering macro


%\useapackage{ragged2e} % for \Centering macro
%\newcolumntype{C}{>{\Centering\arraybackslash}X}
%}
 


\DeclareMathOperator{\deepwalk}{deepwalk}
\DeclareMathOperator{\f}{f}
\DeclareMathOperator{\spath}{nx.shortest-path-length}
\DeclareMathOperator{\landmark}{landmark}
\DeclareMathOperator{\err}{Error}
\DeclareMathOperator{\dist}{dist}
\DeclareMathOperator{\relu}{ReLU}
\DeclareMathOperator{\D}{Dist}
\DeclareMathOperator{\DEC}{DEC}
\DeclareMathOperator{\ENC}{ENC}
\DeclareMathOperator{\SP}{SP}

\newcommand\Tstrut{\rule{0pt}{2.6ex}}         % = `top' strut
\newcommand\Bstrut{\rule[-0.9ex]{0pt}{0pt}} 

\usetheme[pageofpages=of,% String used between the current page and the
                         % total page count.
          alternativetitlepage=true,% Use the fancy title page.
          %titlepagelogo=fim2,% Logo for the first page.
          ]{Torino}
\usecolortheme{freewilly}


\makeatletter
\newbox\@backgroundblock
\newenvironment{backgroundblock}[2]{%
  \global\setbox\@backgroundblock=\vbox\bgroup%
    \unvbox\@backgroundblock%
    \vbox to0pt\bgroup\vskip#2\hbox to0pt\bgroup\hskip#1\relax%
}{\egroup\egroup\egroup}
\addtobeamertemplate{background}{\box\@backgroundblock}{}
\makeatother

\author{Your names}
\title{ \textbf{A Unified Framework for Graph Embedding}}


\institute{Data Science Lab \Tstrut \\ University of Passau}
\date{February, 2020}

% The log drawn in the upper right corner.
\logo{\includegraphics[scale=0.03]{fim2}}
%[height=0.18\paperheight]


\begin{document}

\begin{frame}[t,plain]
\titlepage
\end{frame}

\begin{frame}{Outline}
  \tableofcontents
\end{frame}


\AtBeginSection[]
{
  \begin{frame}<beamer>
    %\frametitle{Outline for section \thesection}
    \tableofcontents[currentsection]
  \end{frame}
}





\section{Introduction}





\begin{frame}{Graph Embedding}

 \begin{center}
    \vspace{0.1cm}
    \includegraphics[scale=0.45]{./pics/embd.pdf}
  \end{center} 
  


\[ \Theta: V \mapsto \mathbb{R}^d \]

\end{frame}


\section{Approach}
\begin{frame}{node2vec}

 use textblock to locate images 

\begin{textblock*}{\paperwidth}(-0.14 \paperwidth, -28 mm)%
\hfill \includegraphics [width=0.4 \linewidth ]{pics/data}%
\end{textblock*}


 you can use pause 
\pause 

\begin{textblock*}{\paperwidth}(-0.30 \paperwidth, 47 mm)%
\hfill \small example for textblock: $k $

\end{textblock*}
 


\end{frame}


  
\begin{frame}{WalkLets}
 
 
\end{frame}
 

 \begin{frame}{HARP}
 
 
\end{frame}

\begin{frame}{struc2vec}
 
 
\end{frame}




\section{Experimental Results}
\begin{frame}{Datasets}

\begin{table}

\caption{\small Statistics of datasets.}
\label{tab:data}
\begin{center}
\small

\begin{tabular}{ccccc}
{\bf Dataset}  & {\bf Nodes } &{\bf Edges } &{\bf Attributes }&{\bf Labels }\Tstrut\\
 \hline 
 \Tstrut
Cora         & 2,708 &  5,429&  1,433 & 7\Tstrut\\
Citeseer      &3,312& 4,660 & 3,703 &  6\Tstrut\\

\end{tabular}
\end{center}
\end{table}

\begin{itemize}
\item Cora~\cite{citeseer} and Citeseer~\cite{citeseer}:
\begin{itemize}
\item The labels indicate publications topics.
\item Attributes are binary representations of words in the corresponding publications.
\end{itemize}


\end{itemize}

\end{frame}




\begin{frame}{Link Prediction}

explain about Link Prediction\\
use textblock to adjust images or tables in the frame

\begin{textblock*}{\paperwidth}(-0.30 \paperwidth, 4 mm)%
\hfill \small example for textblock: $k $

\end{textblock*}


\end{frame}


\begin{frame}{ Node Classification}

explain about node classification and compare the baselines\\
use textblock to adjust images or tables in the frame

\begin{textblock*}{\paperwidth}(-0.30 \paperwidth, 4 mm)%
\hfill \small example for textblock: $k $

\end{textblock*}

\end{frame}


\begin{frame}{ Node Classification}
\tiny

\begin{table}

\caption{\scriptsize Node classification performance (Macro-F1 score) of different methods on different datasets.}


\begin{center}
\begin{tabular}{cccc ccc ccc cc}
\hline
\textbf{Dataset}&
\textbf{Method}&
\multicolumn{10}{c}{\textbf{Macro-F1} } \Tstrut\\
  \hline  
\Tstrut
\Tstrut
  &  &   $10\%$  & $20\%$ & $30\%$  &$40\%$ & $50\%$ & $60\%$ & $70\%$ & $80\%$ &$90\%$ \Tstrut\\

 \multirow{5}{*}{Cora}&baseline & \textbf{0.828} &\textbf{0.841 }&\textbf{0.854 }&\textbf{0.869} &\textbf{0.883}&\textbf{0.901}& \textbf{0.909}& \textbf{0.916}& \textbf{0.921}\Tstrut\\
   & baseline  & 0.663 &  0.673 & 0.684 &0.691  &0.726&  0.754&0.769 & 0.788 &0.808\Tstrut\\
   & baseline  &0.733 &0.752 &0.768 &0.773 &0.788& 0.794& 0.806 &0.814 & 0.822\Tstrut\\
   & baseline  & 0.778 & 0.795 & 0.812&0.822 & 0.837 & 0.854 & 0.861& 0.869 & 0.877 \Tstrut\\
   &  baseline & 0.695& 0.713& 0.729 & 0.732& 0.746& 0.767& 0.788 &0.792 & 0.806\Tstrut\\
     \hline 
     
      \Bstrut
  \multirow{5}{*}{Citeseer}&baseline  &\textbf{0.731} &\textbf{0.739}& \textbf{0.755}& \textbf{0.778}& \textbf{0.786} &\textbf{0.790} &\textbf{0.796} &\textbf{0.804} &\textbf{0.812 }\Tstrut\\
   &  baseline  & 0.538& 0.588 &0.607& 0.610 &0.616 & 0.621 &0.635& 0.656 & 0.677   & \Tstrut\\
    &  baseline  & 0.577 & 0.606&  0.613 & 0.619&  0.628&  0.632&  0.638 & 0.641&  0.642\Tstrut\\
  &  baseline  &0.604  &0.633 &0.671 & 0.678 &0.696& 0.705& 0.723& 0.735& 0.745 \Tstrut\\
   &  baseline  &0.556& 0.571 & 0.614& 0.650 & 0.656 & 0.662  &0.670 &0.666& 0.682\Tstrut\\ 
   \Tstrut\\
    
  \hline 
\end{tabular}
\label{tab:partial}
\end{center}

\end{table}

\end{frame}


\begin{frame}{ Cora Visualization}

use textbloack to adjust images in the frame

\begin{textblock*}{\paperwidth}(-0.30 \paperwidth, 4 mm)%
\hfill \small example for textblock: $k $

\end{textblock*}


\end{frame}




\section{Conclusion}
\begin{frame}{Conclusion}
\begin{itemize}

\item Graph embedding is useful for ...

\item We learned ...
\end{itemize}


\end{frame}


\section{References}
%\begin{frame}[allowframebreaks]{References}

\begin{frame}{References}
\begin{thebibliography}{9}

\tiny

   \bibitem{citeseer} Prithviraj Sen, Galileo Namata, Mustafa Bilgic, Lise Getoor, Brian Gal-
ligher, and Tina Eliassi-Rad, "Collective classification in network data",
AI magazine, 29(3), 93-93, (2008).


\bibitem{fb} Jure Leskovec and Julian J Mcauley, "Learning to discover social circles in ego networks", in Advances in neural information processing
systems, pp. 539-547, (2012).


.
 
 


 
 
\end{thebibliography}
\end{frame}

\begin{frame}{}

\begin{center}
\Huge Thanks for your attention!
\end{center}
\end{frame}


\end{document}

